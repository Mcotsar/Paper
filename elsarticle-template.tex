
\documentclass[review]{elsarticle}
\usepackage{hyperref,lineno}
\usepackage{xcolor}
\modulolinenumbers[5]

\newcommand{\memo}[2]{\textcolor{#1}{#2}}
\newcommand{\xavi}[1]{\memo{orange}{xavi: #1\\}}
\newcommand{\maria}[1]{\memo{blue}{maria: #1\\}}
\journal{Journal of \LaTeX\ Templates}

%%%%%%%%%%%%%%%%%%%%%%%
%% Elsevier bibliography styles
%%%%%%%%%%%%%%%%%%%%%%%
%% To change the style, put a % in front of the second line of the current style and
%% remove the % from the second line of the style you would like to use.
%%%%%%%%%%%%%%%%%%%%%%%

%% Numbered
%\bibliographystyle{model1-num-names}

%% Numbered without titles
%\bibliographystyle{model1a-num-names}

%% Harvard
\bibliographystyle{model2-names.bst}\biboptions{authoryear}

%% Vancouver numbered
%\usepackage{numcompress}\bibliographystyle{model3-num-names}

%% Vancouver name/year
%\usepackage{numcompress}\bibliographystyle{model4-names}\biboptions{authoryear}

%% APA style
%\bibliographystyle{model5-names}\biboptions{authoryear}

%% AMA style
%\usepackage{numcompress}\bibliographystyle{model6-num-names}

%% `Elsevier LaTeX' style
%\bibliographystyle{elsarticle-num}
%%%%%%%%%%%%%%%%%%%%%%%

\begin{document}

\begin{frontmatter}

\title{Another glass of failure? \tnoteref{mytitlenote}}
\tnotetext[mytitlenote]{Fully documented templates are available in the elsarticle package on \href{http://www.ctan.org/tex-archive/macros/latex/contrib/elsarticle}{CTAN}.}

%% Group authors per affiliation:
\author{Mar\'ia Coto-Sarmiento\fnref{myfootnote)}}
\author{Xavier Rubio-Campillo\fnref{myfootnote}}
\address{Radarweg 29, Amsterdam}
\fntext[myfootnote]{Since 1880.}

%% or include affiliations in footnotes:
\author[mymainaddress,mysecondaryaddress]{Elsevier Inc}
\ead[url]{www.elsevier.com}

\author[mysecondaryaddress]{Global Customer Service\corref{mycorrespondingauthor}}
\cortext[mycorrespondingauthor]{Corresponding author}
\ead{support@elsevier.com}

\address[mymainaddress]{1600 John F Kennedy Boulevard, Philadelphia}
\address[mysecondaryaddress]{360 Park Avenue South, New York}

\begin{abstract}
This template helps you to create a properly formatted \LaTeX\ manuscript.
\end{abstract}

\begin{keyword}
Cultural Evolution, Roman Empire, amphorae production, social learning
\end{keyword}

\end{frontmatter}

\linenumbers

%\section{The Elsevier article class}

%\paragraph{Installation} If the document class \emph{elsarticle} is not available on your computer, you can download and install the system package \emph{texlive-publishers} (Linux) or the \LaTeX\ package \emph{elsarticle} using the package manager of your \TeX\ installation, which is typically \TeX\ Live or Mik\TeX.

%\paragraph{Usage} Once the package is properly installed, you can use the document class \emph{elsarticle} to create a manuscript. Please make sure that your manuscript follows the guidelines in the Guide for Authors of the relevant journal. It is not necessary to typeset your manuscript in exactly the same way as an article, unless you are submitting to a camera-ready copy (CRC) journal.

%\paragraph{Functionality} The Elsevier article class is based on the standard article class and supports almost all of the functionality of that class. In addition, it features commands and options to format the
%\begin{itemize}
%\item document style
%\item baselineskip
%\item front matter
%\item keywords and MSC codes
%\item theorems, definitions and proofs
%\item lables of enumerations
%\item citation style and labeling.
%\end{itemize}

%\section{Front matter}
%The author names and affiliations could be formatted in two ways:
%\begin{enumerate}[(1)]
%\item Group the authors per affiliation.
%\item Use footnotes to indicate the affiliations.
%\end{enumerate}
%See the front matter of this document for examples. You are recommended to conform your choice to the journal you are submitting to.


\section{Introduction}

\xavi{Comentario general: por que cada frase la has puesto en un paragrafo distinto? Cada paragrafo deberia ser un tema y no una frase. Y deberias usar el concepto de topic sentence (la primera frase del paragrafo resume el paragrafo).}

Material culture variability allows to understand a part of the mechanism of the human behaviour \citep{basalla1988evolution}. Dynamic of changes on the material culture have been analized by the study of cultural pattern which varying over time and space \citep{eerkens_jelmer_cultural_2007, lycett_cultural_2015} %incluir bibliografía mesoudi y schillinger jas, mesoudi, etc lycett)

\xavi{Esto va al JAS no? La primera frase deberia tener un tono mas arqueologico.}

In the case of archaeology, the detection of visible differences in artefact production in the archaeological records could also explain whether these change are produced by cultural reasons or not based on economical, political and social changes %(meter mesoudi con shilinger other paper).

\xavi{que quieres decir con visible? Y como lo haces visible? Citation needed}


Artefact techniques are socially transmitted by the interaction among individuals. \xavi{que son artefact techniques? Ademas, creo que esta intro es demasiado general...los lectores del JAS ya saben que la gente aprende a hacer ceramica no?}
This mode of learning transmission and several external conditions might affect directly or indirectly the pattern of manufacturing of the artefacts. Different information are shared by social learning generating an accumulation of knowledges which are transmitted from generation to generations in different context and content conditions  \citep{eerkens_jelmer_cultural_2005, neff1992ceramics,henrich_evolution_2003, boyd_cultural_2011}. 

\xavi{Esta frase no dice realmente nada no? Debes ser mas concreta y concisa en lo que quieres transmitir. Dando la vuelta al tema, podrias estructurar el mismo contenido asi: 1. la cultural material que los arqueologos encuentran esta basada siempre en social learning, 2. la arqueologia evolutva analiza el proceso para saber que tipo de social learning se esta usando, 3. la mayoria de estudios se centran en ceramica hecha a mano y variaciones estilisticas 4. creemos que este framework es aplicable a otros contextos y objetivos, como entender los procesos masivos de produccion de amforas en el imperio romano}

Therefore the variation in the material culture will depend on a series of cumulative strategies such as knowledge and interaction of community of producers, velocity of production, technology skills applied, type of transmissions and other factors among others %meter bibliografía (meter cavalli and sforza??)
\xavi{esta frase casi repite la anterior}

%Incluir por el contrario de la tesis (horizontal transmission)
In archaeology, this connection processes have been also influenced by the geographical space related to isolation-by-distance \citep{bjorklund_effect_2010,shennan_isolation-by-distance_2015, van_strien_isolation-by-distance_2015}. Thereby artefact variation might be affected by geographical distance where material culture is more similar in close population who interacted each other. In other case, the correlation between both seems not visible due to different factors \citep{hart_effects_2012}. 
\xavi{no creo que debas introducir aqui isolation-by-distance. Por otra parte, no solo es en arqueologia no? Ni tan siquiera evolucion cultural, sino que es algo general en procesos evolutivos.}

However, different debates revolve around how individuals or groups acquired and transmitted techniques skills and the different modes of transmission   \citep{bowser_learning_2008, mesoudi_cultural_2008, roux_standardization_2015}.In  addition, this challenge is combined with the difficulty detecting the different modes of transmission in the archaeological record.     

%¿debería hablar de los diferentes métodos que se han estudiado para medir esta variacion de materiales? empezando así: 

%In particular, culture evolution provides a set of methods that can be used to account these dynamic of changes, focused on the evolution of the making techniques processes. 

%(autocomentario)método measuring?? (lyman and obrien, mesoudi shillinger gandon y laroux que hablan de CV ) No decoration help us to know. measurebles differences can be appreciated
                              

%más especificamente
This paper explores the changes in the production processes during the Roman Empire. Our study is focused on understanding the pottery-making techniques by analyzing large-scale amphorae production. We study the implication that this production might have on the evolution of social learning of potters. In our case, we can detect measurable differences among this type of amphorae correlated with the geographical distance. 
\xavi{Este es el tema principal del paper; deberia ser explicitado antes y un poco mas detallado: 1. queremos entender los procesos productivos del imperio romano, 2. en concreto, la produccion en masa de anforas, 3. la aplicacion de un framework evolutivo nos permitira identificar que tipo de social learning esta envuelto.}


Specifically, the aim of this study is understanding if pottery-making techniques were transmitted through vertical or horizontal social learning. Our main hypothesis concerns the transmission of the techniques by vertical transmission and how vertical transmission is spreading in time and space. These technological knowledges could have been transmitted from master to disciple and thus continuously. If vertical transmission predominates in this process over horizontal transmission then amphorae made in nearby workshops might share more similar traits than amphorae made from farthest workshop. Otherwise whether horizontal transmission is the main transmission in this process the social learning would be transmitted by workers. Then there would not be differences among workshops on the production.     

\xavi{este parrafo guai, pero es aqui donde deberias hablar de isolation by distance como el concepto por el cual puedes identificar que tipo de social learning esta dominando el proceso}

In our case, the existence of a correlation between spatial distance and morphometric variation will be tested by observing. In this work we have explored the social learning processes associated with amphorae production through a combination of empirical analysis and multivariate methods. 

\xavi{by observing que? Por otra parte, el ultimo parrafo de la intro siempre resume el contenido del paper (next section will define the case study and existing hypotheses, the third one will deal with the methods...blablabla}

\section{The amphoric production in Roman Baetica}

\maria{he quitado las subsection y he añadido el t\'itulo provisional}

Our case study examines the variation of the amphorae production located in \emph{Baetica} (currently Andalusia, south Spain). During the Roman Empire, this ancient province became an important support for the production and distribution of olive oil to the rest of the Empire from Ist to IIIrd centuries \cite{chic_comercio_2005,millet_anforas_1998, rodriguez_baetican_1998} 

\maria{add una cita m\'as Peacock}


For this reason, a large-scale infrastructure of amphorae production was developed around this area to supply the provinces of the Roman Empire with a huge impact during military campaigns \citep{monfort_britannia_1998}. 

\xavi{Por que citas aqui el tema de las campanyas? No seria mejor mencionar el abastecimiento de Roma y a las legiones? No hace falta que sean campanyas, vaya.}

\emph{Baetica} had also a strong connection thought rivers that allowed developing an important trade network around the Mediterranean \citep{garcia_vargas_enrique_formal_2010}. \xavi{mediterranean? Si esta chapter book (al que le falta las pp) va del atlantico no?}
\maria{Porque el comercio se desarrolla principalmente en el mediterraneo.  Por otra parte he puesto fourthcoming porque todavía no se sabe el número de páginas}

As result of this increase, more than 80 pottery workshops were currently located along the Guadalquivir river and its tributaries. (Fig)\xavi{Ande estan las figuras?} \maria{no las he puesto aun pero las he dejado para saber luego donde ponerlas :P} However all the sites examined have currently experimented multiples geographic transformations due to the anthropic action and the dynamic of the rivers (Enrique y Remesal)\xavi{CITATION NEEDED. Ademas, por que es relevante aqui esto de la antropizacion del territorio?}.
\maria{es relevante porque el paisaje de la baetica cambia bastante desde época romana a la actualidad y me gusta matizarlo} 

The majority of amphorae produced in this area belong to \emph{Dressel 20} divided into different typologies \citep{berni_millet_epigrafianforica_2008, martin-kilcher_romischen_1994}. This amphora type was used mostly to transport olive oil for around 300 years in order to satisfy the demand within Roman Empire \citep{rodriguez_economioleicola_1977}. In particular, olive oil was a significant product frequently related in different aspect of the roman daily life such as consumption, lighting and hygiene %Temin?.
\xavi{CITATION NEEDED}

The importance of this commerce is also showed by the fact that Dressel 20 amphorae production were identified with different marks about its provenance although the meaning of the actors in this process seems not clear (CITATION NEEDED). In any case, our main question will be related to understand how the amphorae workshops were organized in \textit{Baetica} area and whether it is possible to identify amphorae made in different workshops. 
Thus, this amphorae production was a particular example of production strategy that experimented few changes around three centuries.   


\maria{FALTA COMPLETAR FINAL}


\xavi{faltaria acabar esta seccion con una discusion del problema que quieres resolver, incluyendo la pregunta de como se organizaban los talleres, que hipotesis hay y como liga esto con lo que has explicado de social learning.}

\maria{no he incluido las hipótesis de partida porque ya hablo de ellas en la introduccion, tendria que incluirlas aqui tambien?}

\section{Material and methods}

\xavi{Es mejor si empiezas con el metodo aqui; si no, el lector no entiende de lo que estas hablando.}

We analyse a dataset of 470 amphorae collected from 5 different workshops excavated. The workshops were located in Malpica (Palma del R\'io, C\'ordoba), Cerro del Bel\'en (Palma del R\'io, C\'ordoba) \citep{diaz_trujillo_excavacion_1992}, Parlamento (Sevilla) \citep{garcia_vargas_anforas_2000}, Villaseca (C\'ordoba)\citep{garcia_vargas_enrique_excavacion_????} and Las Delicias (\'Ecija, Sevilla) \citep{fernandez_excavacion_2001,_atelier_2014}. We created a dataset where were selected 80-100 samples of each pottery workshops. The choice of these workshops corresponded to several reasons. Firstly, the workshops were selected from different spaces in order to analyse the production patterns depending on the distance of each workshop. Secondly, the extended chronology of these workshops serves as proxy to examine changes on the variation shape. In our case, the type Dressel 20 did not experimented especially visible changes on the production pattern during three centuries.%citar berni?
Finally, the workshops selected were open excavated and provided a large number of materials.   

\xavi{Mapa?}
\maria{coming soon}

\maria{he incluido esta subsection antigua field method dentro de esta}
Eight different measurements were taken for each amphorae sample of the 5 workshops studied. The measurements were done using different tools: caliber, square and bevel to take the measurements and profile gauge to draw the pottery shapes. \xavi{no necesitas especificar esto...yo quitaria la ultima frase} \maria{y asi? o la quito?}

The measurements were focused on the rim sherds whose fragments were the most preserved on the archaeological sample. In the case of pottery attributes, rim sherds work as an useful indicator of variability. \maria{CITATION NEEDED and explain why}
Moreover, the measurements were divided into exterior diameter, inside diameter, rim height, rim width, shape width, rim inside height, rim width and protruding rim (Fig). The method requires a large sample size and for this reason we focused on rim sherds. Other significative parts such as handles and bases are found in lesser quantities thus compromising the applicability of the method due to small sample size.

In our study, we have selected five variants according with three centuries
(Dressel B: I; Dressel C: I-II; Dressel D: II; Dressel E: III, Dressel F: III) defined by P. Berni \citep{berni_millet_epigrafianforica_2008} and Martin Kilcher \citep{martin-kilcher_romischen_1994}. All of these variants selected were found in excavations from the proper workshops studied in order to avoid some material which can contaminate the sample. For the proposal of this study, the rest of variants were not taken into account from our study by not having enough material for the analysis. 

Specifically in this type of amphorae CITATION NEEDED differences between variants are identified on the rim sherds and handles. These differences 

\maria{EXPANDIR ESTO}
\xavi{Expande esto; estas variantes tienen que ver con datacion relativa? area? Por otra parte, deberias especificar que todas las amforas fueron encontradas en los talleres, y por eso se sabe que fueron fabricadas alli.}


\subsection{Principal Component Analysis}

The sample selected were tested using statistical method such as Principal Component Analysis and Discriminant Analysis to explore these metrical differences on the rim sherds. 

\xavi{Es mas correcto hablar de "reducir la dimensionalidad" en lugar del numero de variables. Por otra parte, no se si hace falta detallar tanto PCA...mira otros papers similares usando PCA y veras que es tan comun que no hace falta definir mucho (quizas solo especifica la primera frase, y que es comun en arqueo como las obras citadas abajo).}
\maria{es curioso porque en los papers que he visto sobre arqueometria ni se paran con esto pero en los de arqueologia si, por eso lo hice, debería reducirlo mas ahora?}

We used Principal Component Analysis (PCA) to reduce the dimensionality of our dataset. This method allows to create a reduced number of \textit{new variables} which contain all the relevant information of the previous variables without losing relevance. The firsts principal components are expressed as the result of the most variance of the all information from the original variables. Moreover the information is expressed as the result of most variation retained in the first principal components \citep{jolliffe_principal_2002, shennan_quantifying_1997}. 
This method is commonly used in archaeology for the study of the variation of material culture \citep{li_crossbows_2014, schillinger_differences_2016} 
In our study, this method allows to transform our measurements into PCs and take the firsts PCA with more variability in the dataset.  

\maria{anadir +biografia y una figura}


\subsection{Discriminant Linear Analysis} 


The performed results with PCA were analysed with Lineal Discriminant Analyse (LDA). LDA was used to find significant differences among workshops by the combination among variables obtained for the first principal components. Unlike the PCA, LDA identifies which variables allow to distinguish or discriminate each group and how many variables are necessary to achieve the best combination as possible. In our case, this method allowed to demonstrate the correlation between spatial distance and distance among workshops. LDA was used to explore a better separate training set from the results of the most relevant principal components. LDA can classify the PCs result of the measurements into different groups.  We also generate a Confusion Matrix (CM) to able of quantifying the degree of confusion and compare the index of similarity among workshops.  CM calculated the probability of success and error of the results. It generates a matrix where higher value are the results of an incorrect classification. The distance generated with the results of DA will be compared with the spatial distance to see if it exists a correlation between morphometric distance and spatial distance. 

As example, this method has been commonly used in archaeometry \citep{charlton_investigating_2012} \maria{meter citas a cascoporro}, and particularly for a similar study about the production pottery in \emph{Tarraconense} \citep{i_martin_alisis_1998}



\section{Results}

Several multivariate methods such as PCA and LDA were used to quantify the technical differences on the pattern production among workshops. 
\maria{he dejado esta frase a modo introductorio y he quitado las subsection} 

The analysis of PCA produces a set of values for each variable observed. Variables show how much variability exist in the dataset grouped by each principal components. The results, indicated in the Table, show most variability in the firsts principal components than the rest (mostrar el que más con el analisis).
The most differences were focused on the %(poner donde más estuvieron enfocadas)
\xavi{Table o Figure?}    

The patterns observed in the first 2 Principal Components were plotted to visualize the degree of variation by isolation among workshops. The results, shown in Fig., suggested than amphorae from closer workshops tend to be more similar than amphorae made in furthest workshops. In particular, the Fig illustrates how the four closest workshops show variation on PC1 (i.e. Belén, Delicias, Villaseca and Malpica) while Parlamento displays a distinctive pattern than the rest of workshops on PC2 values. 
\xavi{que pattern? esto es parte resultados parte discussion, asi que quizas deberias unir las 2 secciones}


Discriminant Analysis was used to analyze the results obtained from PCA. The results of CM showed than workshops with more troubles to be distinguished such Malpica and Bel\'en shared a minor spatial distance than the rest (see Fig). Therefore, similar amphorae making techniques processes were strongly correlated with the spatial distance. 
\xavi{distance matrix?}
\maria{que pacha}

We compared morphometric and spatial distance by performing peer-to-peer analysis between all workshops. We calculated the geographical distance between each site and the distance among pottery measures, calculated using the previous results. (FIG) shows that the pottery distance is strongly correlated with the spatial distance of workshops.


\xavi{Como decia yo quitaria las subsection. Y si hablas de correlacion deberias cuantificarla (lo miramos cuando este el paper en un segundo draft)}


%decir el bug que hay

Thus, the results suggest a variability on the making-techniques processes might depend on the spatial distance.  

\section{Discussion and Conclusion}


Differences on the making techniques processes among workshops were identified using empirical method and multivariate analysis. \xavi{Esta primera frase es redundante (si el lector ha llegado aqui ya lo sabe} The results show the variability is correlated with spatial distance.  

The analysed morphometric traits suggest that the similarity between amphorae decrease with the spatial distance between the workshops where they were produced. As result, amphorae made in nearby workshops with a minor spatial distance share more traits than amphorae made in pottery workshops furthest. In other words, the variability on the making techniques processes between closer workshops was difficult to differentiate. In our case, Malpica and Bel\'en workshops where the geographical proximity are the closest shared more traits in comparison with other workshops (Parlamento and Las Delicias). Thus the probability of interaction between workshops is increasing when the proximity is closest while this likelihood decreases when the possibility of interaction is low. 

We have observed than rivers courses could have affected in the transmission factors. In the case of the commerce, rivers and its tributaries played an important role for the transport of goods. The huge demand within Roman Empire and the good conditions for the loading and unloading of products (concierto aceite romano//berni) might have influenced the mode of transmission due to the continuous contact between workshops. \xavi{no entiendo esta frase aqui. que quieres decir?} 
\maria{quiero decir que el guadalquivir ofrecia buenas condiciones para el intercambio de materiales o productos}

The results suggest also that vertical transmission could be the main cultural mechanism to explain the variability between workshops. The different morphological traits among workshops seem proper of a low contact between potters from others workshops. The evidenced confirms therefore that these techniques traits were transmitted with high fidelity and only with few changes during three centuries. It would mean that the disciples could have remained the making techniques processes in the workshops where they were trained.  

By contrast, horizontal transmission doesn't seem to be the most probable process. The continuous contact between potters from different places had generated a more homogeneity in the technical practises. Workshops were sharing the same production techniques. As result, it would generate a social network where potters with the same social learning level worked in different workshops at the same time. Our result suggest a progressive contact with closer workshop instead. Moreover, the fact that isolation by distance is detected suggests a limited displacement between distant workshops. Thus, vertical transmission would be explained with this observed process. However, the diversity of social learning processes are clearly complex. In other words, the transmission of knowledges between master and disciple did not discard that horizontal transmission played an important role in this process as well. It can be a process where this vertical transmission dominated at first in the same workshops but consequently this transmission would be affected by workers who exchanged ideas or workers moving to other workshops.  

%podría ser conclusión
The combination of empirical analysis with the statistical methods have provided a strong baseline for a better understanding of the amphorae production in the Roman Empire\xavi{Esto es el JAS, no hace falta que vendas los metodos cuantitativos :-)}
\maria{si, pero es un caso particular que yo no he visto en JAS :p o como podría ponerlo} These methods offer also an strong complement to other methods as archaeometry for the characterization of production sites and places of consumption.  

We have identified measurable differences in the techniques by observing and we have tested these particularities using multivariate methods. Our analysis provides an useful baseline for the exploration of the social learning processes related with amphora production in the Roman Empire. Hence, the results have lightened to understand the link between social learning and archaeological evidence in a diversity of scenarios. 

\section{Acknowledgments}

Data were collected and performed and analysed in R version 3.2.4. statistical language and implemented with the package MASS.
\maria{incluir la url con los datos y el código and citation needed}



%\section{Bibliography styles}

%There are various bibliography styles available. You can select the style of your choice in the preamble of this document. These styles are Elsevier styles based on standard styles like Harvard and Vancouver. Please use Bib\TeX\ to generate your bibliography and include DOIs whenever available.

\section*{References}

%\bibliographystyle{apalike}
\bibliography{mybibfile}

\end{document}

\cite
