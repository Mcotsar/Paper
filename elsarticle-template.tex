
\documentclass[review]{elsarticle}
\usepackage{hyperref,lineno}
\modulolinenumbers[5]

\journal{Journal of \LaTeX\ Templates}

%%%%%%%%%%%%%%%%%%%%%%%
%% Elsevier bibliography styles
%%%%%%%%%%%%%%%%%%%%%%%
%% To change the style, put a % in front of the second line of the current style and
%% remove the % from the second line of the style you would like to use.
%%%%%%%%%%%%%%%%%%%%%%%

%% Numbered
%\bibliographystyle{model1-num-names}

%% Numbered without titles
%\bibliographystyle{model1a-num-names}

%% Harvard
\bibliographystyle{model2-names.bst}\biboptions{authoryear}

%% Vancouver numbered
%\usepackage{numcompress}\bibliographystyle{model3-num-names}

%% Vancouver name/year
%\usepackage{numcompress}\bibliographystyle{model4-names}\biboptions{authoryear}

%% APA style
%\bibliographystyle{model5-names}\biboptions{authoryear}

%% AMA style
%\usepackage{numcompress}\bibliographystyle{model6-num-names}

%% `Elsevier LaTeX' style
%\bibliographystyle{elsarticle-num}
%%%%%%%%%%%%%%%%%%%%%%%

\begin{document}

\begin{frontmatter}

\title{Another glass of failure? \tnoteref{mytitlenote}}
\tnotetext[mytitlenote]{Fully documented templates are available in the elsarticle package on \href{http://www.ctan.org/tex-archive/macros/latex/contrib/elsarticle}{CTAN}.}

%% Group authors per affiliation:
\author{Mar\'ia Coto-Sarmiento\fnref{myfootnote)}}
\author{Xavier Rubio-Campillo\fnref{myfootnote}}
\address{Radarweg 29, Amsterdam}
\fntext[myfootnote]{Since 1880.}

%% or include affiliations in footnotes:
\author[mymainaddress,mysecondaryaddress]{Elsevier Inc}
\ead[url]{www.elsevier.com}

\author[mysecondaryaddress]{Global Customer Service\corref{mycorrespondingauthor}}
\cortext[mycorrespondingauthor]{Corresponding author}
\ead{support@elsevier.com}

\address[mymainaddress]{1600 John F Kennedy Boulevard, Philadelphia}
\address[mysecondaryaddress]{360 Park Avenue South, New York}

\begin{abstract}
This template helps you to create a properly formatted \LaTeX\ manuscript.
\end{abstract}

\begin{keyword}
\texttt{elsarticle.cls}\sep \LaTeX\sep Elsevier \sep template
\MSC[2010] 00-01\sep  99-00
\end{keyword}

\end{frontmatter}

\linenumbers

\section{The Elsevier article class}

\paragraph{Installation} If the document class \emph{elsarticle} is not available on your computer, you can download and install the system package \emph{texlive-publishers} (Linux) or the \LaTeX\ package \emph{elsarticle} using the package manager of your \TeX\ installation, which is typically \TeX\ Live or Mik\TeX.

\paragraph{Usage} Once the package is properly installed, you can use the document class \emph{elsarticle} to create a manuscript. Please make sure that your manuscript follows the guidelines in the Guide for Authors of the relevant journal. It is not necessary to typeset your manuscript in exactly the same way as an article, unless you are submitting to a camera-ready copy (CRC) journal.

\paragraph{Functionality} The Elsevier article class is based on the standard article class and supports almost all of the functionality of that class. In addition, it features commands and options to format the
\begin{itemize}
\item document style
\item baselineskip
\item front matter
\item keywords and MSC codes
\item theorems, definitions and proofs
\item lables of enumerations
\item citation style and labeling.
\end{itemize}

\section{Front matter}

The author names and affiliations could be formatted in two ways:
\begin{enumerate}[(1)]
\item Group the authors per affiliation.
\item Use footnotes to indicate the affiliations.
\end{enumerate}
See the front matter of this document for examples. You are recommended to conform your choice to the journal you are submitting to.



\section{Introduction}


Material culture variability allows to understand a part of the mechanism of the human behaviour \citep{basalla1988evolution}(citar boyd and richerson, el libro de cambios, paper de recombinación). Dynamic of changes on the material culture have been analized by the study of cultural pattern which varying over time and space \citep{eerkens_jelmer_cultural_2007, lycett_cultural_2015} incluir bibliografía mesoudi y schillinger jas, mesoudi, etc lycett).

In the case of archaeology, the detection of visible differences in artefact production in the archaeological record could also explain whether these change are produced by cultural reasons or not based on economical, political and social changes (meter mesoudi con shilinger other paper).

Artefact techniques are socially transmitted by the interaction among individuals.
This mode of learning transmission and several external conditions might affect directly or indirectly the pattern of manufacturing of the artefacts. Different information are shared by social learning generating an accumulation of knowledges which are transmitted from generation to generations in different context conditions  \citep{neff1992ceramics,henrich_evolution_2003, boyd_cultural_2011}. Therefore the variation in the material culture will depend on a series of cumulative strategies such as knowledge of producers, velocity of production, technology skills applied, type of transmissions (leer boyd and richerson) or interaction among communities among others. (poner bibliografía). 

However, different debates revolve around how individuals or groups acquired and transmitted techniques skills and the different modes of transmission   \citep{bowser_learning_2008, mesoudi_cultural_2008, roux_standardization_2015}.     


%Incluir por el contrario de la tesis (horizontal transmission)

In archaeology, this connection processes have been also influenced by the space related to isolation-by-distance \citep{bjorklund_effect_2010,shennan_isolation-by-distance_2015, van_strien_isolation-by-distance_2015}. Thereby artefact variation might be affected by geographical distance where material culture is more similar in close population who interacted each other. In other case, the correlation between both seems not visible due to different factors \citep{hart_effects_2012}



Ceramic evolution is the result of different dynamics of knowdledge?? thought history. 
   

%hablar de los diferentes metedos qeu se han estudiado para medir esta variacion de materiales

In particular, culture evolution provides a set of methods that can be used to account these dynamic of changes, focused on the evolution of the making techniques processes. 

%método measuring?? (lyman and obrien, mesoudi shillinger gandon y laroux que hablan de CV )



%meter concepto de transmisssion vinculada a social learning (incluir francesa)

%meter concepto de isolation by distance 



reflected                                 

%incluir un however


who was the potter and how was the modes of  transmission how this production has a supervivencia of 3 siglos without changes. No decoration help us to know. measurebles differences can be appreciated (meter aqui la roux y mesoudi)




%más especificamente
This paper explores the changes in the production processes during the Roman Empire. Our study is focused on understanding the pottery-making techniques by analyzing large-scale amphorae production. We study the implication that this production might have on the evolution of social learning of potters. In our case, we can detect measurable differences among this type of amphorae correlated with the geographical distance. 

Specifically, the aim of this study is understanding if pottery-making techniques were transmitted through vertical or horizontal social learning. Our main hypothesis concerns the transmission of the techniques by vertical transmission and how vertical transmission is spreading in time and space. These technological knowledges could have been transmitted from master to disciple and thus continuously. If vertical transmission predominates in this process over horizontal transmission then amphorae made in nearby workshops might share more similar traits than amphorae made from farthest workshop, following the isolation-by distance.    

In our case, the existence of a correlation between spatial distance and morphometric variation will be tested by observing. In this work we have explored the social learning processes associated with amphorae production through a combination of empirical analysis and theoretical exploration.



\section{Background}

\subsection{Study area}

Our principal case study examines the variation of the amphorae production located in \emph{Baetica} (currently Andalusia, south Spain). During the Roman Empire, this ancient province became an important support for the production and distribution of  olive oil to the rest of the Empire from Ist to IIIrd centuries \cite{rodriguez_baetican_1998}. For this reason, a large-scale infrastructure of amphorae production was developed around this area to supply the provinces of the Roman Empire with an important impact during military campaigns (remesal concierto, monfort, temin y kessler).

\emph{Baetica} had also a strong connection thought rivers that allowed developing an important trade network around the Mediterranean (Remesal, Vargas). More than 80 pottery workshops were currently located along the Guadalquivir river and its tributaries (citar Berni, Remesalin and Enriquito)(Fig). However all the sites examined have currently experimented multiples geographic transformations due to the anthropic action and the dynamic of the rivers (Enrique y Remesal). 

%Most of workshops were located on the margen izquierdo of the guadalquivir...

The majority of amphorae identified in this area belong to \emph{Dressel 20} typology divided into different sub-typologies (Martin-Kilcher, Berni bibliografia). This amphora type was used mostly to transport olive oil for around 300 years in order to satisfy the demand within Roman Empire (Remesal ingles). In particular, olive oil was a significant product frequently related in different aspect of the roman daily life such as consumption, lighting and hygiene (Temin?).


%hablar de un comercio diferente, que no ha experimentado cambios a lo largo de tres siglos?? del comercio en esa zona de aceite

%The high demand is also showed by the fact that amphorae Dressel 20 were identified with several marks about its provenance (remesal and xavi). However, the meaning of the stamps seems not clear: it could be an agent identified as a olive oil producer or an agent identified as a potter factory. In any case, this paper will be only focused on the study of evolution of the amphorae (Rubio, Remesal sellos). 

\section{Material and methods}

We analyse a dataset of 470 amphorae collected from 5 different workshops excavated. The workshops were located in Malpica (Palma del R\'io, C\'ordoba), Cerro del Bel\'en (Palma del R\'io, C\'ordoba) \citep{diaz_trujillo_excavacion_1992}, Parlamento (Sevilla) \citep{garcia_vargas_anforas_2000}, Villaseca (C\'ordoba) and Las Delicias (\'Ecija, Sevilla) \citep{fernandez_excavacion_2001,_atelier_2014} . We created a dataset where were selected 80-100 samples of each pottery workshops. The choice of these workshops corresponded to several reasons. Firstly, the workshops were selected from different spaces in order to analyse the production patterns depending on the distance of each workshop. Secondly, the extended chronology of these workshops serves as proxy to examine changes on the variation shape. Finally, the workshops selected were open excavated and provided a large number of materials.   


%(hablar de que no han cambiado en tres siglos??)


\subsection{Field methods}


Eight different measurements were taken for each amphorae sample of the 5 workshops studied. The measurements were done by one person using different tools: caliber, square and bevel to take the measurements and profile gauge for drawing the pottery shapes. 

The measurements were focused on the rim sherds whose fragments were the most preserved on the archaeological sample. In the case of pottery attributes, rim sherds work as an useful indicator of variability. Moreover, the measurements were divided into exterior diameter, inside diameter, rim height, rim width, shape width, rim inside height, rim width and protruding rim (Fig). We excluded other measurements such as handles and bases from our study due to the lack of these in the sample.    

In our study, we have selected five variants according with three centuries (Dressel B: I; Dressel C: I-II; Dressel D: II; Dressel E: III, Dressel F: III) defined by P. Berni (bibliografía de Berni) and Martin Kilcher (Martin Kilcher). Specifically, Differences between variants are identified on the rim sherds and handles. For the proposal of this study, the rest of variants were not taken into account from our study by not having enough material for the analysis. 

Finally, the sample selected were tested using statistical method such as Principal Component Analysis and Discriminant Analysis to explore these metrical differences on the rim sherds. 
 
%varianza de clasificaciones que hay? 

\subsection{Principal Component Analysis}

%%PCA (podría poner biblio de Shennan y el de Jollife (2002) de Principal Component Analysis)

We used Principal Component Analysis (PCA) to simplify a large number of variables into a smaller number of variables. This method allows to create a reduced number of \textit{new variables} which contain all the relevant information of the previous variables without losing relevance. The firsts principal components are expressed as the result of the most variance of the all information from the original variables. Moreover the information is expressed as the result of most variation retained in the first principal components \citep{jolliffe_principal_2002, shennan_quantifying_1997}. 
This method is commonly used in archaeology for the study of the variation of material culture \citep{li_crossbows_2014, schillinger_differences_2016} 
In our study, this method allowed us to reduce our dataset of 8 measurements as variables into 2 variables. 

%EXPLICAR UN POCO EL POR QUÉ, NO MIARMA? decir que este meteodo se ha usado en otros caso de arqueología y citar algunos como el de terracotas

\subsection{Discriminant Linear Analysis} 


The performed results with PCA were analysed with Lineal Discriminant Analyse (LDA). LDA was used to find significant differences among workshops by the combination among variables obtained for the first principal components. In spite of being similar to PCA, LDA identifies which variables allow to distinguish or discriminate each group and how many variables are necessary to achieve the best combination as possible. In our case, LDA was used to explore a better separate training set from the results of the most relevant principal components. 

Data were collected and performed in LibreOffice 4.2.8.2 and analysed in R version 3.2.4. statistical language and implemented with the package MASS. 


\section{Results}

Several multivariate methods such as PCA and DLA were used to quantify the technical differences on the pattern production among workshops. 5 workshops were chosen following criteria described above. 


\subsection{Principal Component Analysis}

The analysis of PCA produces a set of values for each variable observed. Variables show how much variability exist in the dataset grouped by each principal components. The results, indicated in the Table, show most variability in the firsts principal components than the rest (mostrar el que más con el analisis).The most differences were focused on the (poner donde más estuvieron enfocadas)

The patterns observed in the first 2 Principal Components were plotted to visualize the degree of variation by isolation among workshops. The results, shown in Fig., suggested than amphorae from closer workshops tend to be more similar than amphorae made in furthest workshops. In particular, the Fig illustrates how the four closest workshops show variation on PC1 (i.e. Belén, Delicias, Villaseca and Malpica) while Parlamento displays a distinctive pattern than the rest of workshops on PC2 values. 


\subsection{Discriminant Analysis}

%Tengo que decir que uso principales componentes antes que analisis discriminante para hacer un trainning 

Discriminant Analysis was used to perform the results of the PCA. We generate a Confusion Matrix (CM) to quantify the degree of confusion among workshops. CM calculated the probability of success and error of the results. It generates a matrix where higher value are the results of an incorrect classification. 
%completar el por qué

The results of Confusion Matrix showed than workshops with more troubles to be distinguished such Malpica and Bel\'en shared a minor spatial distance than the rest (see Fig). Therefore, similar amphorae making techniques processes are strongly correlated with the spatial distance. 



\subsection{Correlation between distance and morphometric}

We compared morphometric and spatial distance by performing peer-to-peer analysis between all workshops. We calculated the geographical distance between each site and the distance among pottery measures, calculated using the previous results. (FIG) shows that the pottery distance is correlated with the spatial distance of workshops.

%Desarrollar y decir cual está más cerca de quien y el bug que hay



Thus, the results suggest an variability on the making-techniques processes correlated with the spatial distance. 

\section{Discussion and Conclusion}


Differences on the making techniques processes among workshops were identified using empirical methods and multivariate analysis. The results show the variability could be affected by the distance. The analysed morphometric traits suggest that the similarity between amphorae decrease with the spatial distance between the workshops where they were produced. As result, amphorae made in nearby workshops with a minor spatial distance share more traits than amphorae made in pottery workshops furthest. In other words, the variability on the making techniques processes between closer workshops were difficult to differentiate. In our case, Malpica and Bel\'en workshops where the geographical proximity are the closest shared more traits in comparison with other workshops (Parlamento and Las Delicias). Thus the probability of interaction between workshops is increasing when the proximity is closest while this likelihood is decreasing when the possibility of interaction is low. 



%incluir ríos

We have observed than rivers courses could have affected in this aspect of transmission. In the case of the commerce, rivers and its tributaries played an important role for the transport of goods. The huge demand within Roman Empire and the good conditions for the loading and unloading of products (concierto aceite romano//berni) might have influenced the mode of transmission by the continuous contact between workshops. 


 


%transmission and social learning

 The results confirm also that vertical transmission could be the main cultural mechanism to explain the variability between workshops. The different morphological traits among workshops seems proper of a low contact between potters from others workshops . The evidenced more probable confirms that pottery techniques would be learned from master to disciple by social learning. These techniques traits, therefore, were transmitted with high fidelity and only with few changes during three centuries. 
 
 
 
 It means that the disciples could have remained the making techniques processes in the workshops where they were trained.  

%(explicar un poco y explicar por trabajadores del mismo nivel workers by the same level).
 



By contrast, horizontal transmission doesn't seem to be the most probable process, although we don't discard its important role in this process (meter remesal con los discipulos). 


\section{Conclusion}
%podría ser conclusión
The combination of empirical analysis with the statistical methods have provided a strong baseline for a better understanding of the amphorae production in the Roman Empire. These methods offer also an strong alternative to other invasive methods  such as archaeometry, etc... We have identified measurable differences in the techniques 

Hence, the results have lightened 



Isolation by distance is supported with our results. 


\section{Bibliography styles}

There are various bibliography styles available. You can select the style of your choice in the preamble of this document. These styles are Elsevier styles based on standard styles like Harvard and Vancouver. Please use Bib\TeX\ to generate your bibliography and include DOIs whenever available.

Here are two sample references: \citep{mesoudi_cultural_2015}

\section*{References}

%\bibliographystyle{apalike}
\bibliography{mybibfile}

\end{document}

\cite